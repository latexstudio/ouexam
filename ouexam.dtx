\def\filedate{15 April 1999}
\def\fileversion{1.0}
% \iffalse meta-comment
%%
%% ouexam.dtx Copyright 1999 Nigel Stanger
%% 
%% You may use this package freely, and also distribute it
%% provided that you don't change it, make any money off
%% it or pretend that you wrote it.
%%
%<*driver>
\documentclass{article}
\usepackage{doc}
\DisableCrossrefs
\CodelineNumbered
\title{The \textsf{ouexam} document class, v\fileversion}
\author{Nigel Stanger\\nstanger@infoscience.otago.ac.nz}
\date{\filedate}
\begin{document}
\maketitle
\DocInput{ouexam.dtx}
\end{document}
%</driver>
%
% \fi
%
%% \CheckSum{339}
%%
%% \CharacterTable
%%  {Upper-case    \A\B\C\D\E\F\G\H\I\J\K\L\M\N\O\P\Q\R\S\T\U\V\W\X\Y\Z
%%   Lower-case    \a\b\c\d\e\f\g\h\i\j\k\l\m\n\o\p\q\r\s\t\u\v\w\x\y\z
%%   Digits        \0\1\2\3\4\5\6\7\8\9
%%   Exclamation   \!     Double quote  \"     Hash (number) \#
%%   Dollar        \$     Percent       \%     Ampersand     \&
%%   Acute accent  \'     Left paren    \(     Right paren   \)
%%   Asterisk      \*     Plus          \+     Comma         \,
%%   Minus         \-     Point         \.     Solidus       \/
%%   Colon         \:     Semicolon     \;     Less than     \<
%%   Equals        \=     Greater than  \>     Question mark \?
%%   Commercial at \@     Left bracket  \[     Backslash     \\
%%   Right bracket \]     Circumflex    \^     Underscore    \_
%%   Grave accent  \`     Left brace    \{     Vertical bar  \|
%%   Right brace   \}     Tilde         \~}
%%
%
% \MakeShortVerb{\|}
%
% \begin{abstract}
% This document class allows you to create Otago University final
% examination papers using \LaTeXe\@. It implements the major formatting
% requirements specified by the University and also provides some useful
% macros to ease the process of building the title page and automatically
% deals with fiddly little issues such as question and section numbering,
% the number of pages in the paper and printing `\textbf{TURN OVER}' in the
% bottom right corner of every page except the last one.
% \end{abstract}
%
% \section{Introduction}
%
% \subsection{Class options}
%
% This document class is based on the \textsf{article} class and accepts any of
% the options accepted by \textsf{article}. The default options are
% \textsf{onecolumn}, \textsf{oneside}, \textsf{a4paper} and \textsf{12pt}.
% Normally the only one you might want to change is the last one.
% 
% \subsection{Required packages}
% 
% This class currently requires the \textsf{calc} and \textsf{ifthen}
% packages. I may look at removing the requirement for these in a
% later version, as neither of them are that essential.
% 
% \subsection{Page margins}
% 
% This class assumes A4 paper. You will probably get weird results if you try
% to do anything different. The margins are set up as follows: top and bottom
% 2cm (headers and footers are inside this margin), left and right 2.54cm (1in).
% 
% \subsection{Page styles}
% 
% There are three main page styles in this class:
% \begin{description}
% 	\item[\textsf{plain}] \DescribeMacro{plain} This is a slight modification of
% the \textsf{plain} page style from \textsf{article}. It produces pages that
% have the page number centered at the top, the paper number in the top right
% corner and the text `\textbf{TURN OVER}' in the bottom right. This is the
% default page style.
% 	\item[\textsf{lastpage}] \DescribeMacro{lastpage} This is the same as
% \textsf{plain}, but without the `\textbf{TURN OVER}'. This is used for the
% last page of the examination. You normally will not have to use this
% yourself --- the class should take care of it for you. I have managed to
% confuse it in the past, however, so there may be odd occasions where you
% have to set the page style of the last page manually. It will be fairly
% obvious when you need to do this.
% 	\item[\textsf{titlepage}] \DescribeMacro{titlepage} This is the same as
% \textsf{plain}, but without the page header, and is used for the title page.
% Again, you will normally not use this yourself, as the |\maketitlepage|
% macro handles this for you (see section~\ref{Sec:TitlePage}).
% \end{description}
% 
% \section{Writing the paper}
% 
% \subsection{Sections}
% 
% Examination papers may have multiple sections, `numbered' A, B, \ldots.
% Rather than redefine the existing section macros, this class defines a new
% macro \DescribeMacro{\newsection} |\newsection| that generates a new
% section. This macro has one optional parameter which is an instruction or
% instructions for answering the questions in the section. The default is
% `ANSWER ALL QUESTIONS'. Every section begins on a new page, and the section
% title is formatted as `\textbf{\underline{Section A}}', in |\large| size.
% 
% \subsection{Questions}
% 
% The \DescribeEnv{questions} \textsf{questions}, \DescribeEnv{subquestions}
% \textsf{subquestions} and \DescribeEnv{subsubquestions}
% \textsf{subsubquestions} environments are used to write questions. These
% three environments are actually just defined to be identical to the
% \textsf{enumerate} environment, and work in the same manner. The labels for
% list items have been redefined so that questions are numbered according to
% University formatting requirements.
% 
% Questions must be numbered sequentially throughout the entire paper
% regardless of section boundaries. Since it is quite possible to have
% multiple \textsf{questions} environments within a document, the class
% defines the \DescribeMacro{question} |question| counter, which is
% global to the entire document and not reset every time a new
% \textsf{questions} environment is started, unlike the |enumi|
% counter. The |enumii| and |enumiii| counters are unchanged.
% 
% Each part of a question is worth some number of marks, which needs to be
% displayed somewhere after the question. You can specify the number of marks
% for a question using the \DescribeMacro{\marks} |\marks| macro, which prints
% `(\emph{n} marks)' right-justified on the line. It has a single parameter,
% the number of marks. The macro automatically determines whether to print
% `mark' or `marks' as appropriate. Other than right-justifying, the macro
% does no other formatting or positioning; how you position the marks relative
% to a question is up to you.
% 
% If you have a question with several sub-parts, each worth some number of
% marks, you should display the total number of marks for the question at the
% end of the question. You can do this with the \DescribeMacro{\totalmarks}
% |\totalmarks| macro, which prints `\textbf{[TOTAL \emph{n} MARKS]}
% right-justifed on the line. It has a single parameter, the number of marks
% (that is, it does not add up the total for you!). As with |\marks|, it
% determines whether to print `MARK' or `MARKS' automatically, and does no
% other formatting or positioning.
% 
% \subsection{Generating the title page}
% \label{Sec:TitlePage}
% 
% This class defines a collection of macros that let you fill in the various
% parts of the examination title page. This is analogous to the process you
% use to generate the title of a document in, for example, the
% \textsf{article} class (using |\title|, |\author|, |\date| and
% |\maketitle|).
% 
% The \DescribeMacro{\examyear} |\examyear| macro lets you specify the year
% in which the examination is being held, for example, |\examyear{1999}|. This
% macro is required. I may look at defaulting this to the current year at some
% later stage.
% 
% The \DescribeMacro{\department} |\department| macro lets you specify the
% name of the department that produced the examination paper, for example,
% |\department{Information| |Science}|. This macro is required.
% 
% The \DescribeMacro{\papernumber} |\papernumber| macro lets you specify the
% paper number that the examination is for, for example,
% |\papernumber{INFO 211}|. This macro is required.
% 
% The \DescribeMacro{\papertitle} |\papertitle| macro lets you specify the
% title of the paper, for example,
% |\papertitle{Systems Analysis and Design Methods}|. This macro is required.
% 
% Some papers are offered in both semesters. The \DescribeMacro{\semester}
% |\semester| macro lets you specify which semester the examination is for.
% The parameter can be either `1' or `2', for example, |\semester{2}|. This
% macro is optional --- if you omit it, no semester information is generated.
% 
% The \DescribeMacro{\timeallowed} |\timeallowed| macro lets you specify the
% length of the examination in hours, for example, |\timeallowed{2}|. This
% macro is optional --- if you omit it, it defaults to three (3) hours.
% 
% If calculators are allowed in the examination, use the
% \DescribeMacro{\allowcalculators} |\allowcalculators| macro. This macro is
% optional --- if you omit it, a sentence is printed saying that calculators
% are \emph{not} allowed.
% 
% The \DescribeMacro{\instructions} |\instructions| macro lets you specify
% instructions on how candidates should complete the
% examination, for example, |\instructions{Answer ALL| |questions.}|. This macro
% is optional. Note that the content of the instructions can be just about
% anything. It is up to you to control formatting, such as how you want lines
% broken, etc. This also true of the |\material|, |\copiesof| and
% |\otherinstructions| macros described below.
% 
% The \DescribeMacro{\material} |\material| macro lets you specify any
% additional material that candidates are provided in addition to the
% examination paper itself, for example, |\material{SQL schema definition}|.
% This macro is optional.
% 
% The \DescribeMacro{\copiesof} |\copiesof| macro lets you specify any
% material that candidates are allowed to bring into the examination, for
% example, |\copiesof{McFadden &| |Hoffer, 5th edition}|. This macro is
% optional.
% 
% The \DescribeMacro{\otherinstructions} |\otherinstructions| macro lets you
% specify any other instructions not covered by any of the above, for example,
% |\otherinstructions{No smoking| |allowed}|. This macro is optional.
% 
% Once you have specified the content of the title page using the above
% macros, you simply issue a \DescribeMacro{\maketitlepage} |\maketitlepage|
% to generate the front page of the examination (cf. |\maketitle|). The title
% page generated meets University formatting requirements.
% 
% Note that the number of pages in the examination paper is generated
% and inserted into the title page automatically --- you do not need to specify
% it manually.
%
% \section{Example}
% 
% The code shown below is for a subset of the INFO212 1998 final examination.
% The title page produced by this code is shown in figure~\ref{Fig.Output1},
% and the second page of the examination is shown in figure~\ref{Fig.Output2}.
% Particular points to note are:
% \begin{itemize}
% 	\item This examination has no sections and only one \textsf{questions}
% 	  environment, but the usual procedure would be to end any current
% 	  \textsf{questions} environment, then issue a |\newsection|.
% 	\item The number of hours is not specified, so it defaults to three.
% 	\item This exam allows calculators.
% 	\item The parameter of the |\instructions|, |\material|, |\copiesof| and
% 	  |\other| |instructions| macros can be just about anything, as illustrated
% 	  below by the |\material| macro.
% 	\item Formatting within questions is totally under user control, for
% 	  example, each question in this examination is terminated with a
% 	  linebreak (|\\|) to produce some space between questions.
% \end{itemize}
% 
% \begin{verbatim}
% \documentclass[12pt]{ouexam}
% 
% \examyear{1998}
% \department{INFORMATION SCIENCE}
% \papernumber{INFO 212}
% \papertitle{Database Design and Management Principles}
% \instructions{Answer \underline{ALL} questions.}
% \material{Happy Joe SQL schema  \\
%           Oracle Rdb SQL syntax diagrams for:  \\
%           \hspace*{1cm} CREATE TRIGGER  \\
%           \hspace*{1cm} CREATE VIEW  }
% \allowcalculators
% 
% \begin{document}
% 
% \maketitlepage
% 
% \underline{ANSWER ALL QUESTIONS}  \\[\baselineskip]
% 
% \begin{questions}
%   \item Study the Happy Joe schema supplied with this paper
%         (see pages~\pageref{start.schema}--\pageref{end.schema})
%         and answer the following questions.
%   \begin{subquestions}
%     \item Explain fully in English the purpose of the following items
%         in the schema (see page~\pageref{end.schema}):
%     \begin{subsubquestions}
%       \item Trigger \textbf{T1}.           \marks{2}  \\
%       \item Trigger \textbf{T2}.           \marks{1}  \\
%       \item View \textbf{V1}.              \marks{1}  \\
%       \item View \textbf{V2}.              \marks{2}  \\
%       \item View \textbf{V3}.              \marks{2}  \\
%       \item Stored procedure \textbf{P1}.  \marks{3}  \\
%     \end{subsubquestions}
%     
%     \item Write SQL code to define a single view that will show all
%           cars that have travelled more than 80,000 kilometres (or
%           50,000 miles for older cars), and which have been sold for
%           less than their purchase price. (Note: the odometers of older
%           cars use miles instead of kilometres. The attribute miles\_km
%           in the Car table contains `K' if the odometer reading is in
%           kilometres or `M' if it is in miles.)  \\
% 
%           The syntax diagram for the CREATE VIEW statement has been
%           supplied with this paper (see page~\pageref{page.syntax}).  \\
%           \marks{5}  \\
% 
%     \item ...  \\
%   \end{subquestions}
%   \totalmarks{20}
%     .
%     .
%     .
% \end{questions}
% 
% \end{document}
% \end{verbatim}
% \normalsize
% 
% \begin{figure}
% 	\fbox{%
% 		\begin{minipage}{\textwidth}
%			\vspace*{1cm}
% 			\begin{center}
% 				{\large \textbf{UNIVERSITY OF OTAGO EXAMINATIONS 1998}}
% 				\\[0.625\baselineskip]
% 				\fbox{%
% 				  \framebox[9.9cm]{%
% 					\begin{tabular}{c}
% 					  \\
% 					  {INFORMATION SCIENCE}  \\
% 					  \\
% 					  {Paper INFO 212}  \\
% 					  \\
% 					  {Database Design and Management Principles}  \\
% 					  \\
% 					\end{tabular}%
% 				  }%
% 				}
% 				\mbox{}\\[0.625\baselineskip]
% 				{\small \textbf{(TIME ALLOWED: 3 HOURS)}}
% 				\\[0.625\baselineskip]
% 			\end{center}
% 			\small
% 			\hspace*{1cm}\underline{This examination paper comprises 9 pages}
% 			\\[0.625\baselineskip]
% 			\hspace*{1cm}\underline{Candidates should answer questions as %
% 			                        follows:}
% 			\\[0.625\baselineskip]
% 			\hspace*{1.625cm}
% 			\begin{minipage}{8.4cm}
% 				Answer \underline{ALL} questions.
% 			\end{minipage}
% 			\\[0.625\baselineskip]
% 			\hspace*{1cm}\underline{The following material is provided:}
% 			\\[0.625\baselineskip]
% 			\hspace*{1.625cm}
% 			\begin{minipage}{8.4cm}
% 				Happy Joe SQL schema	\\
% 				Oracle Rdb SQL syntax diagrams for:	\\
% 				\hspace*{0.625cm} CREATE TRIGGER	\\
% 				\hspace*{0.625cm} CREATE VIEW
% 			\end{minipage}
% 			\\[0.625\baselineskip]
% 			\hspace*{1cm}\underline{Use of calculators:}
% 			\\[0.625\baselineskip]
% 			\hspace*{1.625cm}
% 			\begin{minipage}{8.4cm}
% 				Electronic calculators may be used in this examination.
% 				\\[0.625\baselineskip]
% 				Please note: where the use of calculators is permitted, these
% 				must be non-pro\-gram\-ma\-ble, hand-held calculators without
% 				alphabetic key pads.
% 			\end{minipage}
% 			\\[0.625\baselineskip]
% 			\mbox{}\hfill(Subject to inspection by the examiners.)\hspace*{1cm}
% 			\\[0.625\baselineskip]
% 			\hspace*{1cm}\underline{Candidates are permitted copies of:}
% 			\\[0.625\baselineskip]
% 			\hspace*{1.625cm}
% 			\begin{minipage}{8.4cm}
% 			\end{minipage}
% 			\\[0.625\baselineskip]
% 			\mbox{}\hfill(Subject to inspection by the examiners.)\hspace*{1cm}
% 			\\[0.625\baselineskip]
% 			\hspace*{1cm}\underline{Other Instructions:}
% 			\\[0.625\baselineskip]
% 			\hspace*{1.625cm}
% 			\begin{minipage}{8.4cm}
% 			\end{minipage}
% 			\\[0.75cm]
% 			\mbox{}\hfill\textbf{TURN OVER}\hspace*{1cm}
% 			\vspace*{1cm}
% 		\end{minipage}
% 	}
% 	\caption{Title page produced by \textsf{ouexam}.}
% 	\label{Fig.Output1}
% \end{figure}
% 
% \begin{figure}
% 	\fbox{%
% 		\begin{minipage}{\textwidth}
% 			\small
% 			\vspace*{1cm}
% 			\hspace*{1cm}\hfill 2 \hfill INFO 211 \hspace*{1cm}
% 			\\[0.625\baselineskip]
% 			\hspace*{1cm}\underline{ANSWER ALL QUESTIONS}
% 			\\[1cm]
% 			\hspace*{0.8cm}
% 			\begin{minipage}{10cm}
% 			\renewcommand{\labelenumiii}{(\theenumiii)}
% 			\begin{enumerate}
% 				\item Study the Happy Joe schema supplied with this paper (see
% 				   pages~6--8) and answer the following questions.
% 				\begin{enumerate}
% 					\item Explain fully in English the purpose of the following
% 					   items in the schema (see page~9):
% 					\begin{enumerate}
% 						\item Trigger \textbf{T1}. \hfill (2 marks)
% 						\item Trigger \textbf{T2}. \hfill (1 mark)
% 						\item View \textbf{V1}. \hfill (1 mark)
% 						\item View \textbf{V2}. \hfill (2 marks)
% 						\item View \textbf{V3}. \hfill (2 marks)
% 						\item Stored procedure \textbf{P1}. \hfill (3 marks)
% 						\\[-0.5\baselineskip]
% 					\end{enumerate}
% 					
% 					\item Write SQL code to define a single view that will show
% 					  all cars that have travelled more than 80,000 kilometres
% 					  (or 50,000 miles for older cars), and which have been sold
% 					  for less than their purchase price. (Note: the odometers
% 					  of older cars use miles instead of kilometres. The
% 					  attribute miles\_km in the Car table contains `K' if the
% 					  odometer reading is in kilometres or `M' if it is in
% 					  miles.)	\\[-0.5\baselineskip]
% 			
% 					  The syntax diagram for the CREATE VIEW statement has been
% 					  supplied with this paper (see page~9).	\\
% 					  \mbox{}\hfill (5 marks)	\\[-0.5\baselineskip]
% 				
% 					\item When a feature code is changed in the Feature table,
% 					  this change should be automatically cascaded to the
% 					  Car\_feature table. Write SQL code to define a trigger
% 					  that does this.	\\[-0.5\baselineskip]
% 			
% 					  The syntax diagram for the CREATE TRIGGER statement has
% 					  been supplied with this paper (see page~9).	\\
% 					  \mbox{}\hfill (4 marks)	\\[-0.5\baselineskip]
% 				\end{enumerate}
% 				\mbox{}\hfill \textbf{[TOTAL 20 MARKS]}
% 				\\[1cm]
% 				\mbox{}\hfill\textbf{TURN OVER}
% 				\vspace*{1cm}
% 			\end{enumerate}
% 			\end{minipage}
% 		\end{minipage}
% 	}
% 	\caption{Second page of the examination paper.}
% 	\label{Fig.Output2}
% \end{figure}
% 
% 
% \StopEventually{}
% 
% \section{The code}
% 
% \subsection{Preamble}
% 
% This document class is based on the \textsf{article} class and accepts any
% of the options accepted by \textsf{article}. The only `native' option to
% \textsf{ouexam} is \textsf{multichoice}; the rest are passed through to
% \textsf{article}.
%    \begin{macrocode}
\NeedsTeXFormat{LaTeX2e}[1998/06/01]
\ProvidesClass{ouexam}%
  [1999/04/15 Otago University examination paper class]
\DeclareOption*{\PassOptionsToClass{\CurrentOption}{article}}
\ProcessOptions
\LoadClass[onecolumn,oneside,a4paper,12pt]{article}
%    \end{macrocode}
% 
% 
% \subsection{Required packages}
% 
% The required packages are \textsf{calc} and \textsf{ifthen}, but these may
% change.
%    \begin{macrocode}
\RequirePackage{calc}
\RequirePackage{ifthen}
%    \end{macrocode}
%
% 
% \subsection{Page setup}
% 
% \begin{macro}{\oddsidemargin}
% \begin{macro}{\topmargin}
% \begin{macro}{\textwidth}
% \begin{macro}{\textheight}
% A4 paper is assumed (since this class is intended only for Otago examination
% papers, this seems a reasonable assumption). The margins are top and bottom
% 2cm, left and right 2.54cm (1in). This is the only place where the
% \textsf{calc} package is used, and could easily be rewritten to not use it.
%    \begin{macrocode}
\setlength{\oddsidemargin}{0cm}
\setlength{\topmargin}{-0.54cm}
\setlength{\textwidth}{15.9cm}
\setlength{\textheight}{25.7cm-\headheight-\headsep-\footskip}
%    \end{macrocode}
% \end{macro}
% \end{macro}
% \end{macro}
% \end{macro}
% 
% \subsection{Page styles}
% 
% \begin{macro}{\ps@plain}
% The \textsf{plain} page style is the same as that in the \textsf{article}
% class, except that it produces pages with the page number centered at the
% top, the paper number in the top right corner and the text `\textbf{TURN
% OVER}' in the bottom right:
%    \begin{macrocode}
\renewcommand{\ps@plain}{%
    \renewcommand{\@oddhead}{\hfil\textrm{\thepage}\hfil\@pnumber}%
    \renewcommand{\@evenhead}{\@oddhead}%
    \renewcommand{\@oddfoot}{\hfil\textbf{TURN OVER}}%
    \renewcommand{\@evenfoot}{\@oddfoot}%
  }
%    \end{macrocode}
% \end{macro}
% \textsf{plain} is the default page style.
%    \begin{macrocode}
\pagestyle{plain}
%    \end{macrocode}
% 
% \begin{macro}{\ps@lastpage}
% The \textsf{lastpage} page style is the same as \textsf{plain}, but without
% the `\textbf{TURN OVER}':
%    \begin{macrocode}
\newcommand{\ps@lastpage}{%
    \renewcommand{\@oddhead}{\hfil\textrm{\thepage}\hfil\@pnumber}%
    \renewcommand{\@evenhead}{\@oddhead}%
    \renewcommand{\@oddfoot}{}%
    \renewcommand{\@evenfoot}{\@oddfoot}%
  }
%    \end{macrocode}
% \end{macro}
% This page style should be automatically applied to the last page of the
% document.
%    \begin{macrocode}
\AtEndDocument{\thispagestyle{lastpage}}
%    \end{macrocode}
% 
% \begin{macro}{\ps@titlepage}
% The \textsf{titlepage} page style is the same as \textsf{plain}, but without
% the header:
%    \begin{macrocode}
\newcommand{\ps@titlepage}{%
    \renewcommand{\@oddhead}{}%
    \renewcommand{\@evenhead}{\@oddhead}%
    \renewcommand{\@oddfoot}{\hfil\textbf{TURN OVER}}%
    \renewcommand{\@evenfoot}{\@oddfoot}%
  }
%    \end{macrocode}
% \end{macro}
% 
% \subsection{Environments}
% 
% \begin{environment}{questions}
% \begin{environment}{subquestions}
% \begin{environment}{subsubquestions}
% The \textsf{questions}, \textsf{subsubquestions} and
% \textsf{subsubquestions} environments are just redefinitions of the
% \textsf{enumerate} environment:
%    \begin{macrocode}
\newenvironment{questions}{\begin{enumerate}}{\end{enumerate}}
\newenvironment{subquestions}{\begin{enumerate}}{\end{enumerate}}
\newenvironment{subsubquestions}{\begin{enumerate}}{\end{enumerate}}
%    \end{macrocode}
% \end{environment}
% \end{environment}
% \end{environment}
% 
% 
% \begin{macro}{question}
% We cannot use the |enumi| counter for the \textsf{questions} environment
% because |enumi| will get reset every time we start a new \textsf{questions}
% environment. We therefore need a separate counter:
%    \begin{macrocode}
\newcounter{question}
%    \end{macrocode}
% \end{macro}
% 
% \begin{macro}{\labelenumi}
% \begin{macro}{\p@enumii}
% \begin{macro}{\p@enumiii}
% \begin{macro}{\labelenumiii}
% The University requires that questions be numbered as `1.', `2.', \ldots,
% sub-questions be numbered as `(a)', `(b)', \ldots, and sub-sub-questions be
% numbered as `(i)', `(ii)', \ldots. We therefore need to redefine the
% label-generating macros slightly. In particular, we need to redefine
% |\labelenumi| so that it uses the new |question| counter.
%    \begin{macrocode}
\renewcommand{\labelenumi}{%
    \refstepcounter{question}
    \arabic{question}.
  }
\renewcommand{\p@enumii}{}
\renewcommand{\p@enumiii}{(\theenumii)}
\renewcommand{\labelenumiii}{(\roman{enumiii})}
%    \end{macrocode}
% \end{macro}
% \end{macro}
% \end{macro}
% \end{macro}
% 
% \subsection{Miscellaneous macros}
% 
% \begin{macro}{\newsection}
% The |\newsection| macro generates a new examination section, appropriately
% numbered. The optional parameter contains instructions for answering the
% questions in the new section, and defaults to `ANSWER ALL QUESTIONS'.
%    \begin{macrocode}
\newcommand{\newsection}[1][ANSWER ALL QUESTIONS.]{%
%    \end{macrocode}
% Every section begins on a new page.
%    \begin{macrocode}
    \newpage
%    \end{macrocode}
% The section title is formatted as `\textbf{\underline{Section A}}', in
% |\large| size. Note that sections are `numbered' A, B, \ldots.
%    \begin{macrocode}
    \refstepcounter{section}
    {\large\noindent\textbf{\underline{Section~\Alph{section}}}}
    \\[0.5\baselineskip]
    #1
  }
%    \end{macrocode}
% \end{macro}
% 
% \begin{macro}{\marks}
% The |\marks| macro specifies the number of marks for a question, and prints
% it right-justified on the line as `(\emph{n} marks)', where \emph{n} is
% passed as a parameter to the macro. It uses \textsf{ifthenelse} to determine
% whether to print `mark' or `marks', based on the value of the parameter.
%    \begin{macrocode}
\newcommand{\marks}[1]{%
    \mbox{}\hfill
    (#1~\ifthenelse{#1=1}{mark}{marks})
  }
%    \end{macrocode}
% \end{macro}
% 
% \begin{macro}{\totalmarks}
% The |\totalmarks| macro specifies the total marks for a question that
% comprises several sub-parts, and prints it out right-justified on the line
% as `\textbf{[TOTAL \emph{n} MARKS]}, where \emph{n} is passed as a parameter
% to the macro. As with |\marks|, it determines whether to print `MARK' or
% `MARKS' automatically. A future extension might be to calculate the total
% number of marks automatically, but not yet!
%    \begin{macrocode}
\newcommand{\totalmarks}[1]{%
    \mbox{}\hfill
    \textbf{[TOTAL #1~\ifthenelse{#1=1}{MARK}{MARKS}]}
  }
%    \end{macrocode}
% \end{macro}
% 
% \subsection{Title page macros}
% 
% \begin{macro}{\examyear}
% \begin{macro}{\@year}
% The |\examyear| macro specifies the year in which the examination is being
% held. It redefines the |\@year| macro which is used in |\maketitlepage|:
%    \begin{macrocode}
\newcommand{\examyear}{\renewcommand{\@year}}
%    \end{macrocode}
% This macro is required:
%    \begin{macrocode}
\newcommand{\@year}{%
    \ClassError{ouexam}{No {\protect\examyear} given}\@eha
  }
%    \end{macrocode}
% \end{macro}
% \end{macro}
% 
% \begin{macro}{\department}
% \begin{macro}{\@dept}
% The |\department| macro specifies the name of the department, and is required.
% It redefines the |\@dept| macro which is used in |\maketitlepage|:
%    \begin{macrocode}
\newcommand{\department}{\renewcommand{\@dept}}
\newcommand{\@dept}{%
    \ClassError{ouexam}{No {\protect\department} given}\@eha
  }
%    \end{macrocode}
% \end{macro}
% \end{macro}
% 
% \begin{macro}{\papernumber}
% \begin{macro}{\@pnumber}
% The |\papernumber| macro specifies the paper number, and is required.
% It redefines the |\@pnumber| macro which is used in |\maketitlepage|:
%    \begin{macrocode}
\newcommand{\papernumber}{\renewcommand{\@pnumber}}
\newcommand{\@pnumber}{%
    \ClassError{ouexam}{No {\protect\papernumber} given}\@eha
  }
%    \end{macrocode}
% \end{macro}
% \end{macro}
% 
% \begin{macro}{\papertitle}
% \begin{macro}{\@ptitle}
% The |\papertitle| macro specifies the title of the paper, and is required.
% It redefines the |\@ptitle| macro which is used in |\maketitlepage|:
%    \begin{macrocode}
\newcommand{\papertitle}{\renewcommand{\@ptitle}}
\newcommand{\@ptitle}{%
    \ClassError{ouexam}{No {\protect\papertitle} given}\@eha
  }
%    \end{macrocode}
% \end{macro}
% \end{macro}
% 
% \begin{macro}{\semester}
% \begin{macro}{\@semester}
% The |\semester| macro specifies which semester the examination is for. Legal
% values for the parameter are `1' or `2'. This macro uses \textsf{ifthenelse}.
% It redefines the |\@semester| macro which is used in |\maketitlepage|:
%    \begin{macrocode}
\newcommand{\semester}[1]{%
    \ifthenelse{#1=1}%
      {\renewcommand{\@semester}{Semester One  \\}}%
      {\renewcommand{\@semester}{Semester Two  \\}}
  }
%    \end{macrocode}
% |\semester| is optional --- if you omit it, |\@semester| remains empty:
%    \begin{macrocode}
\newcommand{\@semester}{}
%    \end{macrocode}
% \end{macro}
% \end{macro}
% 
% \begin{macro}{\timeallowed}
% \begin{macro}{\@hours}
% The |\timeallowed| macro specifies the length of the examination in hours.
% It redefines the |\@hours| macro which is used in |\maketitlepage|:
%    \begin{macrocode}
\newcommand{\timeallowed}{\renewcommand{\@hours}}
%    \end{macrocode}
% |\timeallowed| is optional --- if you omit it, |\@hours| defaults to three:
%    \begin{macrocode}
\newcommand{\@hours}{3}
%    \end{macrocode}
% \end{macro}
% \end{macro}
% 
% There is no macro to specify the number of pages in the examination because
% it is done automatically by inserting a |\newlabel| that refers to the last
% page directly into the |.aux| file:
%    \begin{macrocode}
\AtEndDocument{%
    \immediate\write\@auxout{%
        \string\newlabel{page.last}{{}{\thepage}}
    }
  }
%    \end{macrocode}
% This label is then referenced in |\maketitlepage|.
% 
% \begin{macro}{\instructions}
% \begin{macro}{\@instructions}
% The |\instructions| macro specifies instructions on how candidates should
% answer questions, and is optional. It redefines the |\@instructions| macro
% which is used in |\maketitlepage|:
%    \begin{macrocode}
\newcommand{\@instructions}{}
\newcommand{\instructions}[1]{\renewcommand{\@instructions}{#1}}
%    \end{macrocode}
% \end{macro}
% \end{macro}
% 
% \begin{macro}{\material}
% \begin{macro}{\@material}
% The |\material| macro lets specifies any additional material that candidates
% are provided in addition to the examination paper itself, and is optional.
% It redefines the |\@material| macro which is used in |\maketitlepage|:
%    \begin{macrocode}
\newcommand{\@material}{}
\newcommand{\material}[1]{\renewcommand{\@material}{#1}}
%    \end{macrocode}
% \end{macro}
% \end{macro}
% 
% \begin{macro}{\allowcalculators}
% \begin{macro}{\@calculators}
% The |\allowcalculators| macro specifies whether or not calculators are
% allowed in the examination. If |\allowcalculators| is omitted, |\@calculators|
% defaults to `may not':
%    \begin{macrocode}
\newcommand{\@calculators}{may not}
%    \end{macrocode}
% Issuing |\allowcalculators| changes the redefines |\@calculators| to `may':
%    \begin{macrocode}
\newcommand{\allowcalculators}{\renewcommand{\@calculators}{may}}
%    \end{macrocode}
% |\@calculators| is used in |\maketitlepage|.
% \end{macro}
% \end{macro}
% 
% \begin{macro}{\copiesof}
% \begin{macro}{\@copiesof}
% The |\copiesof| macro specifies any material that candidates are allowed to
% bring into the examination, and is optional. It redefines the |\@copiesof|
% macro which is used in |\maketitlepage|:
%    \begin{macrocode}
\newcommand{\@copiesof}{}
\newcommand{\copiesof}[1]{\renewcommand{\@copiesof}{#1}}
%    \end{macrocode}
% \end{macro}
% \end{macro}
% 
% \begin{macro}{\otherinstructions}
% \begin{macro}{\@otherinst}
% The |\otherinstructions| macro specifies any other instructions not covered
% by any of the above, and is optional. It redefines the |\@otherinst|
% macro which is used in |\maketitlepage|:
%    \begin{macrocode}
\newcommand{\@otherinst}{}
\newcommand{\otherinstructions}[1]{\renewcommand{\@otherinst}{#1}}
%    \end{macrocode}
% \end{macro}
% \end{macro}
%
% \begin{macro}{\maketitlepage}
% The |\maketitlepage| macro is the \textsf{ouexam} analogue of |\maketitle|.
% It takes all of the information provided by the above macros and generates a
% properly formatted examination title page.
%    \begin{macrocode}
\newcommand{\maketitlepage}{%
  \thispagestyle{titlepage}
%    \end{macrocode}
% This header information for the examination is printed centered at the top
% of the page. The department, paper number and paper title (and the semester
% information if required) are placed inside a double box.
%    \begin{macrocode}
  \begin{center}
    {\Large \textbf{UNIVERSITY OF OTAGO EXAMINATIONS \@year}}
    \\[\baselineskip]
    \fbox{%
      \framebox[\linewidth]{%
        \begin{tabular}{c}
          \\
          {\large \@dept}  \\
          \\
          {\large Paper \@pnumber}  \\
          \\
          {\large \@ptitle}  \\
          \@semester  \\
        \end{tabular}%
      }%
    }
    \mbox{}\\[\baselineskip]
%    \end{macrocode}
% Next, the time allowed for completing the examination:
%    \begin{macrocode}
    \textbf{(TIME ALLOWED: \@hours\ HOURS)}
    \\[\baselineskip]
  \end{center}
%    \end{macrocode}
% The number of pages is automatically calculated, as descibed earlier. All we
% need to do is reference the |page.last| label that we put into the |.aux|
% file:
%    \begin{macrocode}
  \underline{This examination paper comprises %
             \pageref{page.last} pages}
  \\[\baselineskip]
%    \end{macrocode}
% Print out the instructions for answering questions:
%    \begin{macrocode}
  \underline{Candidates should answer questions as follows:}
  \\[\baselineskip]
  \hspace*{1cm}
  \begin{minipage}{13.5cm}
    \@instructions
  \end{minipage}
  \\[\baselineskip]
%    \end{macrocode}
% Print out any provided material:
%    \begin{macrocode}
  \underline{The following material is provided:}
  \\[\baselineskip]
  \hspace*{1cm}
  \begin{minipage}{13.5cm}
    \@material
  \end{minipage}
  \\[\baselineskip]
%    \end{macrocode}
% Print out use of calculators section:
%    \begin{macrocode}
  \underline{Use of calculators:}
  \\[\baselineskip]
  \hspace*{1cm}
  \begin{minipage}{13.5cm}
    Electronic calculators \@calculators\ be used in this examination.
    \\[\baselineskip]
    Please note: where the use of calculators is permitted, these
    must be non-pro\-gram\-ma\-ble, hand-held calculators without
    alphabetic key pads.
  \end{minipage}
  \\[\baselineskip]
  \mbox{}\hfill (Subject to inspection by the examiners.)
  \\[\baselineskip]
%    \end{macrocode}
% Print out anything that candidates are allowed to bring:
%    \begin{macrocode}
  \underline{Candidates are permitted copies of:}
  \\[\baselineskip]
  \hspace*{1cm}
  \begin{minipage}{13.5cm}
    \@copiesof
  \end{minipage}
  \\[\baselineskip]
  \mbox{}\hfill (Subject to inspection by the examiners.)
  \\[\baselineskip]
%    \end{macrocode}
% Print out other instructions, and finish.
%    \begin{macrocode}
  \underline{Other Instructions:}
  \\[\baselineskip]
  \hspace*{1cm}
  \begin{minipage}{13.5cm}
    \@otherinst
  \end{minipage}
  \\[\baselineskip]
  \newpage
}
%    \end{macrocode}
% \end{macro}
% \Finale
